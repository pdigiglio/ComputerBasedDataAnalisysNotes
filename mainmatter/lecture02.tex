\chapter{Lecture 2}
\section{What is statistics?}

Statistics can be defined in several ways:
\begin{itemize}
	\item
		The science of learning from or making sense out of data;
	\item
		The science of uncertainty;
	\item
		The language of experimental sciences.
\end{itemize}
The founding work was by Jakob Bernoulli~\cite{bernoulli}.

\subsection{Randomness}
Many processes have a random nature as, for example, a coin toss.
Although Classical Physics is in principle capable of predicting the outcome of the toss, the randomness lies in the initial condition which are not known with infinite precision. 


Similarly in the classical description of a gas it's impossible to know the initial conditions for all the particles so a statistical description has to be used.


In Quantum Mechanics, physical laws are \emph{intrinsically} probabilistic thus the randomness is not only due to our lack of knowledge.
Both our ``ignorance'' and the intrinsic randomness lead to uncertainty in the measurement.
What we need is a way to quantify uncertainties using probability concepts.


\section{Some basics concepts}

An \emph{outcome}\index{outcome} is one of the possible result of an experiment and it's unique.
Each trial of an experiment gives exactly one outcome.


The \emph{sample space}\index{sample space} is the set of \emph{all} the possible outcomes.

\Ex{
	We'll now list some sample spaces:
	\begin{itemize}
		\item
			For a coin toss $S=\Set{\text{Head},\text{Tail}}$;
		\item
			For a 6-faces rolling dice $S=\Set{\text{digits on upper surface}} = \Set{1,2,\dots,6}$;
		\item
			For a calorimeter $S=\Set{\text{energy of measured particle}} = [E_\textup{min}, E_\textup{max}] \s\R$.
\end{itemize}}


An \emph{event}\index{event} is a set of outcomes which satisfies certain conditions and for which the probability is assigned: it's a subset of the sample space $S$.
An event happened when the outcome of an experiment is an element of the element.


\Ex{Let's say that the outcome of rolling a dice is $X=2$, then all the events $\Set{2}$, $\Set{X \text{ even}}$, $\Set{X > 1 }$ and so on happened.}


The \emph{event space}\index{event!space} is the set of all the subsets of the sample space.
For a coin toss, for example $\Sigma = \Set{\0,\Set{\text{H}},\Set{\text{T}}, S}$ where $\0$ is the null set corresponding to the \emph{impossible event}\index{event!impossible} and $S$ is the whole sample space i.e.~the \emph{sure event}\index{event!sure}.

\section{Definition of probability}

Here the Kolmogorov axioms (1933) follow.

Consider an event space $\Sigma$ and two element $A,B\in\Sigma$ with underlying sample space $S$, then
\begin{enumerate}
	\item
		$P(A) \in \R$ and $P(A) \ge 0$ for each $A\in\Sigma$ i.e.~for each $A\seq S$;
	\item
		To fix the normalization, the probability of the sure event is $P(S) = 1$;
	\item
		If $A$ and $B$ are \emph{disjoint}\index{disjoint events}, namely $A\cap B = \0$, then $P(A\cup B) = P(A) + P(B)$ is the probability for either $A$ \emph{or} $B$ to happen.
\end{enumerate}
From these axioms, the probability of an event can be interpreted as its area normalized to the area of the whole sample space
\begin{equation}\label{eq:interpretationProbabilityArea}
	P(A) = \frac{\abs{A}}{\abs{S}}.
\end{equation}
Moreover it follows that:
\begin{enumerate}
	\item
		Since $S = S\cup\0$ and $S\cap\0 = \0$ we have $P(\0) + P(S) = P(\0)  +1 = 1$ so $P(\0) = 0$; 
	\item\label{point:two}
		For each $A\in\Sigma$ the space $S$ can be built up as $S = A \cup \bar{A}$ where $\bar{A}$ is $A$'s complement into $S$ and $ A \cap \bar{A} = \0$ so $P(\bar{A}) = 1 - P(A)$.
	\item
		From the point~\ref{point:two}, since every probability is greater or equal to zero, it follows that for each $A\in\Sigma$ we have $0\le P(A) \le 1$;
	\item
		If $A \seq B$ then $B = A\cup (B\sm A)$ and $A\cap (B\sm A)=\0$ so $P(A)\le P(A) + P(B\sm A) = P(B)$;
	\item
		If $A\cap B \neq \0$ then $A\cup B$ can be written as $A \cup(B\sm A)$.
		In this way the sets $A$ and $B\sm A$ are disjoint and their union gives $A\cup B$ so
		\begin{equation}\label{eq:sumProbNotDisjointIntermediate}
			P(A\cup B) = P(A) + P(B\sm A).
		\end{equation}
		Now $B\sm A = B \sm(A\cap B)$ thus $B = (B\sm A) \cup (A\cap B)$ where again $(B\sm A) \cap (A\cap B) = \0$ and 
		$P(B) = P(B\sm A) + P(A\cap B)$.
		We can use this last relation and perform a substitution in the~\eqref{eq:sumProbNotDisjointIntermediate} to obtain
		\begin{equation}
			P(A\cup B) = P(A) + P(B) - P(A\cap B).
		\end{equation}
		Subtracting the probability (i.e.~the area) of $A\cap B$ avoids double-counting the area of the intersection.
\end{enumerate}

\subsection{Conditional probability}
\label{sec:condProb}

We call $P(A\mid B)$ the probability for the event $A\in\Sigma$ to happen \emph{given that} another event $B\in\Sigma$ happened.
This is like constraining the sample space to the event $B$, in fact $P(B\mid B) = 1$ so $B$ now is the sure event
\begin{equation}\label{eq:cond_prob}
	P(A\mid B) =\frac{P(A\cap B)}{P(B)}\quad\text{if } B\neq \0
\end{equation}
if $B = \0$ then $P(A\mid B) = 0$.
\begin{figure}
	\centering
	\begin{venndiagram2sets}[%
			labelNotAB={$S$}
		]
		\fillACapB
	\end{venndiagram2sets}
	\caption{If the probability on $S$ is uniform, the conditional probability~\eqref{eq:cond_prob} is the ratio between the area of the intersection $A\cap B$ and the area of $B$.}
	\label{fig:cond_prob}
\end{figure}
To understand this formula we can refer to figure~\ref{fig:cond_prob}, where all the outcomes of the sample space have the same (uniform) probability, and to the interpretation given by equation~\eqref{eq:interpretationProbabilityArea}.
The probability of $A \cap B$ is the ratio between the area of the gray set and the whole sample space $S$, which is normalized to \num{1}.
To restrict the sample space to the event $B$ we must divide by its relative area with respect to $S$, namely $P(B)$.


From this it follows that $P(A\mid S) = P(A)$ and $P(A\mid B) \ge P(A\cap B)$.




Two events are said to be \emph{statistically independent} if their joint probability can be written as the product of their single probabilities, namely
\begin{equation}
	P(A\cap B) = P(A)\,P(B)
\end{equation}
i.e.~if and only if $P(A\mid B) = P(A)$.
This means that the occurrence of $B$ does not effect the probability of the event $A$ but it \emph{does not} mean the event are disjoint.
However, if $A$ and $B$ are disjoint events their conditional probability vanishes since $A\cap B = \0$.
\Ex{%
	Let us consider $A\coloneqq\Set{6 \text{ in first trial}}$, $B\coloneqq\Set{6 \text{ in second trial}}$ and 
	$C\coloneqq\Set{\text{sum of first and second trial is }8}$ in rolling twice a dice.
	Then the events $A$ and $B$ are independent among each other but $A$ and $C$ are not.%
}

\section{Interpretation of probability}

The previous axiomatic definition of probability gives a mathematical framework but it does not say what these probabilities mean and how to assign them.

\subsection{Frequentist interpretation}

This is an empirical interpretation: the most common.
Given $N$ identical trials of an experiment, be $N_A$ the number of occurrences of the event $A$, then $P(A)$ is defined in terms of the \emph{relative frequency}\index{relative freq.~of occurrence} of occurrence as
\begin{equation}\label{eq:freq_prob}
	P(A)\coloneqq \lim_{N\to\infty}\frac{N_A(N)}{N},
\end{equation}
where the dependence of $N_A$ on $N$ has been explicitly indicated.
In this way, the probability is defined \emph{after} the outcomes for the experiment are known so we can talk about \emph{objective posterior probability}\index{objective posterior probability}.

\subsubsection{Issues}

\begin{itemize}
	\item
		The $\lim_{N\to\infty}$ is mathematically not well-defined: in practice only a finite number of experiments can be performed so the question is how fast the sequence~\eqref{eq:freq_prob} converges?
	\item
		This procedure does \emph{not} define how to assign probabilities to events which are intrinsically not repeatable.
		For instance the probability that the Higgs boson exists or that $\SI{510.997}{\kilo\electronvolt\per \ensuremath{c}\squared}\le m_e \le\SI{510.999}{\kilo\electronvolt\per \ensuremath{c}\squared}$ are not known.
		The probabilities for these events are either \num{1} or \num{0} and they do not depend on  repetition of experiments.
\end{itemize}

Thus in the frequentist approach probabilities are associated only with data, i.e.~repeatable experiments.
So in this framework only statements about data can be made and questions like <<how probable is the theory given the data?>> cannot be answered.

\section{Bayes' theorem}

This theorem relates the conditional probabilities $P(A\mid B)$ and $P(B\mid A)$.
Since $P(A\mid B)\,P(B) = P(A\cap B) = P(B\cap A) = P(B\mid A)\,P(A)$ one gets
\begin{equation}\label{eq:bayes_theorem}
	P(A\mid B) = \frac{P(B\mid A)\,P(A)}{P(B)}.
\end{equation}
The quantity $P(A)$ is called \emph{prior probability}\index{prior probability}.

\subsection{Law of total probability}

Let's partition the sample space $S$ into pairwise disjoint subset (i.e.~events) $A_i$ such that $\cup_{i=1} A_i = S$ and $A_i\cap A_j = \0$ when $i\neq j$.
Then, given $B\in\Sigma$ (i.e.~$B\seq S$)
\begin{equation}
	\begin{aligned}
		P(B)
		&= P(B\cap S) = P(B\cap (\cup_{i=1} A_i) )= P( \cup_{i=1}[B\cap A_i] )\\
  &=\sum_iP(B\cap A_i) - \sum_{i\neq j} P([B\cap A_i]\cap[B\cap A_j])\\
  &=\sum_iP(B\cap A_i) - \sum_{i\neq j} P(B\cap [A_i\cap A_j])\\
  &=\sum_iP(B\cap A_i) - \sum_{i\neq j} P(B\cap \0)\\
  &=\sum_iP(B\cap A_i) =\sum_iP(B\mid A_i)\,P(A_i).
	\end{aligned}
\end{equation}
So the Bayes' theorem~\eqref{eq:bayes_theorem} can be written as 
\begin{equation}
	P(A\mid B) = \frac{P(B\mid A)\,P(A)}{\sum_iP(B\mid A_i)\,P(A_i)}.
\end{equation}
With the simplest partition of $S$, which is $S=A\cup\bar A$, the previous formula reads
\begin{equation}
	P(A\mid B) = \frac{P(B\mid A)\,P(A)}{P(B\mid A)\,P(A) + P(B\mid \bar A)\,P(\bar A)}.
\end{equation}
\Ex{%
	In rolling a dice
	\begin{equation}
		P(3\mid\text{num.~is odd}) = \frac{1/6}{1/2} = \frac{1}{3},
	\end{equation}
	while
	\begin{equation}
		\begin{aligned}
			P(\text{odd}\mid 3)
			&\overset{\text{\eqref{eq:cond_prob}}}{=} \frac{1/6}{1/6} = 1 \\
   &\overset{\text{\eqref{eq:bayes_theorem}}}{=} \frac{P(3\mid \text{odd})\,P(\text{odd})}{P(3)} = \frac{(1/3)(1/2)}{1/6} = 1.
		\end{aligned}
	\end{equation}%
}

\Ex{%
	In virology we need a test for some virus and we know that
	\begin{itemize}
		\item
			The false negative rate is \SI{2}{\percent}, i.e.~the specificity is \SI{98}{\percent};
		\item
			The false positive rate is \SI{5}{\percent}, i.e.~the sensitivity is \SI{95}{\percent};
		\item
			The population $A$ has \num{e4} people and \SI{40}{\percent} of them are infected.
	\end{itemize}
	Let's call $h$ the healthy people and $\bar h$ the  infected ones.

	\begin{figure}[]
		\centering
		\subfloat[][Population $A$.]{%
			\begin{tikzpicture}

\tikzstyle{level 1}=[sibling distance = 30mm]
\tikzstyle{level 2}=[sibling distance = 13mm]

\begin{scope}[
	auto=left,
	every node/.style={
		ellipse,
%		draw,
		inner sep=1.5pt
	},
	level distance= 25mm
]
	\node (start) {Population $A$}
		child {
			node (ph) {\num{.6}}
				child { node (p-h) {\num{.57}} }
				child { node (p+h) {\num{.03}} }
		}
	child {
		node (phb) {\num{.4}}
			child { node (p-hb) {\num{.008}} }
			child { node (p+hb) {\num{.392}} }
	};
\end{scope}

\begin{scope}[
	color=black!60,%
	every node/.style={
		anchor = south,
		sloped
	}
]
	\path (start) -- node {$P(h)$} (ph);
	\path (ph) -- node {$P(-\mid h)$} (p-h);
	\path (ph) -- node [anchor=south] {$P(+\mid h)$} (p+h);

	\path (start) -- node [anchor=south] {$P(\bar{h})$} (phb);

	\path (phb) -- node {$P(-\mid \bar{h})$} (p-hb);
	\path (phb) -- node [anchor=south] {$P(+\mid \bar{h})$} (p+hb);
\end{scope}

\end{tikzpicture}
		}\qquad
		\subfloat[][Population $B$.]{%
			\begin{tikzpicture}

\tikzstyle{level 1}=[sibling distance = 30mm]
\tikzstyle{level 2}=[sibling distance = 13mm]

\begin{scope}[
	auto=left,
	every node/.style={
		ellipse,
%		draw,
		inner sep=1.5pt
	},
	level distance= 25mm
]
	\node (start) {Population $B$}
		child {
			node (ph) {\num{.99}}
				child { node (p-h) {\num{.9405}} }
				child { node (p+h) {\num{.0495}} }
		}
	child {
		node (phb) {\num{.01}}
			child { node (p-hb) {\num{.0002}} }
			child { node (p+hb) {\num{.0098}} }
	};
\end{scope}

\begin{scope}[
	color=black!60,%
	every node/.style={
		anchor = south,
		sloped
	}
]
	\path (start) -- node {$P(h)$} (ph);
	\path (ph) -- node {$P(-\mid h)$} (p-h);
	\path (ph) -- node [anchor=south] {$P(+\mid h)$} (p+h);

	\path (start) -- node [anchor=south] {$P(\bar{h})$} (phb);

	\path (phb) -- node {$P(-\mid \bar{h})$} (p-hb);
	\path (phb) -- node [anchor=south] {$P(+\mid \bar{h})$} (p+hb);
\end{scope}

\end{tikzpicture}
		}
		\caption{}

	\end{figure}


	(finish example)
}


